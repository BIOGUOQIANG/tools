\documentclass[11pt,a4paper,oneside]{article}

\usepackage{amsmath}
\usepackage{graphicx}
\usepackage{epstopdf}
\usepackage{booktabs}    %  fancy tables
\usepackage[T1]{fontenc} % fancy fonts
\usepackage{mathpazo}    % fancy math fonts
%\usepackage{fullpage}
\usepackage[dvips]{geometry}
\usepackage{pdfsync}
\usepackage{subfig}
\usepackage[sort]{natbib}
\usepackage{rotating}
\usepackage{pdflscape}
\usepackage{hyperref}

\renewcommand{\baselinestretch}{1.5} % Line spacing 1.5
\geometry{centering,text={18cm,22cm}}

\clubpenalty = 10000
\widowpenalty = 10000
\setlength{\parskip}{1.2ex}
\setlength{\parindent}{0em}


% Here are a few math abbreviations that may be useful. Note that math 
% variables should generally be set in italic, vectors in bold italic, and
% constants, such as e and i should be upright. Operators like the 'd' in 
% dx  and 'D' in D/Dt should also be upright. 
\newcommand{\ir}{\mathrm{i}}
\newcommand{\dr}{\mathrm{d}}
\newcommand{\D}{\mathrm{D}}
\newcommand{\e}{\, \mathrm{e}} \newcommand{\er}{\mathrm{e}}
\newcommand{\vb}{\bm {v}\xspace}
\newcommand{\ub}{\bm {u}\xspace}

% The following makes ODEs and PDEs easier to write.
% For an example, see the second problem below. - Vallis Solution template
\newcommand{\dd}[3][]{{\dr^{#1} #2 \over \dr #3^{#1}}}
\newcommand{\pp}[3][]{{\partial^{#1} #2 \over \partial #3^{#1}}}
\newcommand{\DD}[1]{{\D#1 \over \D t}}

% Example: \dd \psi x + \pp[2]y x  -> dpsi/dx + d2y/dx2

% Table code
%\begin{table}[h]
%	\centering
%	\begin{tabular}{ccc}
%		\hline
%		 		& 	&  \\ \hline
%				&	& 	\\ 
% 		\hline
%	\end{tabular}
%	\caption{}
%	\label{tab:}
%\end{table}

% Bibliography Code
%\bibliographystyle{elsarticle-harv}
%\bibliography{BIBFILENAME WITHOUT EXT}{}