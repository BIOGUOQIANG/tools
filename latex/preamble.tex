\documentclass[11pt,a4paper,oneside]{article}

\usepackage{graphicx}
\usepackage{color}
\usepackage{epstopdf}
\usepackage{psfrag}
\usepackage{cmap}
\usepackage{booktabs}    %  fancy tables
\usepackage[T1]{fontenc} % better font encoding
\usepackage{mathpazo}    % fancy math fonts
\usepackage[utf8]{inputenc} % unicode input encoding
%\usepackage{fullpage}
\usepackage[dvips]{geometry}
\usepackage{pdfsync}
\usepackage{subfig}
\usepackage[sort]{natbib}
\usepackage{rotating}
\usepackage{pdflscape}
\usepackage{hyperref}
%\usepackage{breakurl} % Only for dvips not pdflatex
%\usepackage{siunitx} % Messes up tables
\usepackage[protrusion=true,expansion]{microtype}
\usepackage{mathtools}

\renewcommand{\baselinestretch}{1.5} % Line spacing 1.5
\geometry{centering,text={18cm,22cm}}

\clubpenalty = 10000
\widowpenalty = 10000
\setlength{\parskip}{1.2ex}
\setlength{\parindent}{0em}


% Here are a few math abbreviations that may be useful. Note that math 
% variables should generally be set in italic, vectors in bold italic, and
% constants, such as e and i should be upright. Operators like the 'd' in 
% dx  and 'D' in D/Dt should also be upright. 
\newcommand{\ir}{\mathrm{i}}
\newcommand{\dr}{\mathrm{d}}
\newcommand{\D}{\mathrm{D}}
\newcommand{\e}{\, \mathrm{e}} \newcommand{\er}{\mathrm{e}}

% The following makes ODEs and PDEs easier to write.
% For an example, see the second problem below. - Vallis Solution template
\newcommand{\dd}[3][]{{\frac{\dr^{#1} #2}{\dr #3^{#1}}}}
\newcommand{\pp}[3][]{{\frac{\partial^{#1} #2}{\partial #3^{#1}}}}
\newcommand{\DD}[1]{{\frac{\D#1}{\D t}}}

% These are some of my math shortcuts
\newcommand{\mb}[1]{\boldsymbol{#1}} % math bold font
\newcommand{\ol}[1]{\overline{#1}}   
\newcommand{\x}{\times}
\newcommand{\mO}{\mathcal{O}}
\newcommand{\ubr}{\underbracket}
\newcommand{\disp}{\displaystyle}
\newcommand{\vb}{\mb {v}\xspace}
\newcommand{\ub}{\mb {u}\xspace}

\newcommand{\textbox}[1]{\centerline{\framebox{#1}}}

% non-dimensional numbers
\newcommand{\Ro}{\mathit{Ro}}
\newcommand{\Ri}{\mathit{Ri}}
\newcommand{\Rif}{\Ri_f}
\newcommand{\Pra}{\mathit{Pr}}
\newcommand{\Pe}{\mathit{Pe}}
\newcommand{\Ra}{\mathit{Ra}}
\newcommand{\Rey}{\mathit{Re}}
\newcommand{\Fr}{\mathit{Fr}}
\newcommand{\Bu}{\mathit{Bu}}


% Example: \dd \psi x + \pp[2]y x  -> dpsi/dx + d2y/dx2

% Table code
%\begin{table}[h]
%	\centering
%	\begin{tabular}{ccc}
%		\hline
%		 		& 	&  \\ \hline
%				&	& 	\\ 
% 		\hline
%	\end{tabular}
%	\caption{}
%	\label{tab:}
%\end{table}

% Bibliography Code
%\bibliographystyle{elsarticle-harv}
%\bibliography{BIBFILENAME WITHOUT EXT}{}